\documentclass{article}
\usepackage{graphicx}
\usepackage{listings}
\usepackage{color}
\usepackage{geometry}
\usepackage{array}
\usepackage[utf8]{inputenc}
\usepackage{amsmath}
%\usepackage{algorithm2e}
%\usepackage[noend]{algpseudocode}
\usepackage{algorithm}% http://ctan.org/pkg/algorithm
\usepackage{algpseudocode}% http://ctan.org/pkg/algorithmicx

\definecolor{dkgreen}{rgb}{0,0.6,0}
\definecolor{gray}{rgb}{0.5,0.5,0.5}
\definecolor{mauve}{rgb}{0.58,0,0.82}
\date{\today}
\author{Tarik Atlaoui}
\lstset{frame=tb,
  language=C,
  aboveskip=3mm,
  belowskip=3mm,
  showstringspaces=false,
  columns=flexible,
  basicstyle={\small\ttfamily},
  numbers=none,
  numberstyle=\tiny\color{gray},
  keywordstyle=\color{blue},
  commentstyle=\color{dkgreen},
  stringstyle=\color{mauve},
  breaklines=true,
  breakatwhitespace=true,
  tabsize=3
}
\begin{document}

\makeatletter
\def\BState{\State\hskip-\ALG@thistlm}
\makeatother

\begin{titlepage}
	\enlargethispage{2cm}
	\newcommand{\HRule}{\rule{\linewidth}{0.5mm}}
	\center
	\textsc{\LARGE
	SORBONNE UNIVERSITÉ 
	} \\[1cm]
	\HRule \\[0.4cm]
	{ \huge \bfseries Réponse au question du projet Moteur d’exécution de workflows \\[0.15cm] }
	\HRule \\[4cm]
	\large{Tarik Atlaoui} \\[3cm]
	05 Mai 2020 \\[3cm]

\end{titlepage}

\textbf{1. Pourquoi ne peut-on pas utiliser le nom de la méthode en tant qu’identifiant de tâche ?}
\newline
\newline
   Si on utiliser le nom d'une méthode en tant qu'identifiant de tâche l'utilisateur ne pourrais pas surcharger les méthodes.
   Car deux méthode avec le meme nom mais des parametre différent représenterais que un seul noeud
\newline
\newline
\textbf{2. Donner les grandes lignes de votre algorithme de la méthode execute. Il est attendu un
algorithme décrit de manière synthétique et non un copier/coller de votre code.}

\begin{algorithm}
  \caption{Algorithm Exercice 3}\label{euclid}
  \begin{algorithmic}[1]
    \Procedure{Execute}{$a,b$}
      \State $G\gets Graph<String>$
      \State $C\gets Context<String , Object>$
      \State $Mtab\gets Method[]$
      \State $result\gets Map<String , Object>$
      \For{\texttt{<node in Graph>}}\Comment{ont execute les method racine et stock}
        \State \texttt{<do stuff>}\Comment{et si elle a des dépendance linkFrom on la stock dans Mtab}      
        \If{$nodeParam$ is in $C$ or $nodeNbParam$ == 0}
          \State $result\gets resultat$
        \EndIf
        \EndFor
      \For{\texttt{<m in Mtab>}}
        \State \texttt{<invoke m en recupérant les parametre dans G et C>}
      \EndFor
      \State \textbf{return} $result$
    \EndProcedure
  \end{algorithmic}
\end{algorithm}

\textbf{3. Décrire comment vous avez assuré le parallélisme des tâches indépendantes.}
\newline
\newline
    Nous avons considére chaque noeud du graphe et lui avons dédier un thread , il commence par chercher la 
    méthode qu'il doit appeler, puis pour chaqu'un de c'est argument essaye de le recupérer , si ce dernier n'existe pas encore 
    il attend une fois que il a fini de recupérer la valeur de c'est arrgument il invoke la fonction et mets le resultat dans la map
\newline
\newline
\textbf{1. Décrire votre protocole de communication}
\newline
\newline
\textbf{2. Justifier votre choix de l’API de communication que vous avez utilisée.}
\newline
\newline
\textbf{3. Quelle hypothèse doit-on faire sur le type des objets du contexte ? Justifiez.}
\newline
\newline
\textbf{4. Quel mécanisme avez-vous utilisé pour la notification d’avancement au client ?}
\newline
\newline
\textbf{1. Quel est le protocole de communications entre maître-esclaves et éventuellement entre
esclave-esclave ?}
\newline
\newline
\textbf{2. Comment le maître affecte ses tâches équitablement sur les esclaves ?}
\newline
\newline
\textbf{3. Comment les esclaves sont assurés de respecter leur borne de tâche courante ?}
\newline
\newline
\textbf{4. Comment gérez-vous la communication d’un résultat entre deux tâches dépendantes ? Don-
ner les avantages et les inconvénients de votre solution.}
\newline
\newline
\textbf{5. (niveau 2) Comment gérez-vous le fait d’avoir plusieurs jobs en cours d’exécution sur le
cluster ?}
\newline
\newline
\textbf{6. (niveau 3) Décrire le mécanisme qui permet de détecter la panne d’un esclave. Comment
gérez-vous la réaffectation des tâches perdues ?}

\end{document}
%\begin{algorithm}
%  \caption{Algorithm Exercice 3}\label{euclid}
%  \begin{algorithmic}[1]
%    \Procedure{Execute}{$a,b$}\Comment{The g.c.d. of a and b}
%      \State $r\gets a\bmod b$
%      \While{$r\not=0$}\Comment{We have the answer if r is 0}
%        \State $a\gets b$
%        \State $b\gets r$
%        \State $r\gets a\bmod b$
%      \EndWhile\label{euclidendwhile}
%      \For{\texttt{<some condition>}}
%        \State \texttt{<do stuff>}
%      \EndFor
%      \State \textbf{return} $b$\Comment{The gcd is b}
%    \EndProcedure
%  \end{algorithmic}
%\end{algorithm}