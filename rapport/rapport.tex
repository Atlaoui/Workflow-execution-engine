\documentclass{article}
\usepackage{graphicx}
\usepackage{listings}
\usepackage{color}
\usepackage{geometry}
\usepackage{array}
\usepackage[utf8]{inputenc}
\usepackage{amsmath}
%\usepackage{algorithm2e}
%\usepackage[noend]{algpseudocode}
\usepackage{algorithm}% http://ctan.org/pkg/algorithm
\usepackage{algpseudocode}% http://ctan.org/pkg/algorithmicx
%https://shantoroy.com/latex/how-to-write-algorithm-in-latex/
\definecolor{dkgreen}{rgb}{0,0.6,0}
\definecolor{gray}{rgb}{0.5,0.5,0.5}
\definecolor{mauve}{rgb}{0.58,0,0.82}
\date{\today}
\author{Tarik Atlaoui}
\lstset{frame=tb,
  language=C,
  aboveskip=3mm,
  belowskip=3mm,
  showstringspaces=false,
  columns=flexible,
  basicstyle={\small\ttfamily},
  numbers=none,
  numberstyle=\tiny\color{gray},
  keywordstyle=\color{blue},
  commentstyle=\color{dkgreen},
  stringstyle=\color{mauve},
  breaklines=true,
  breakatwhitespace=true,
  tabsize=3
}
\begin{document}

\makeatletter
\def\BState{\State\hskip-\ALG@thistlm}
\makeatother

\begin{titlepage}
	\enlargethispage{2cm}
	\newcommand{\HRule}{\rule{\linewidth}{0.5mm}}
	\center
	\textsc{\LARGE
	SORBONNE UNIVERSITÉ 
	} \\[1cm]
	\HRule \\[0.4cm]
	{ \huge \bfseries Réponse au question du projet Moteur d’exécution de workflows \\[0.15cm] }
	\HRule \\[4cm]
	\large{Tarik Atlaoui} \\[3cm]
	03 Juin 2020 \\[3cm]

\end{titlepage}
                                                                                                                         
\textbf{1. Pourquoi ne peut-on pas utiliser le nom de la méthode en tant qu’identifiant de tâche ?}
\newline
\newline
Si on utilisait le nom d'une méthode en tant qu'identifiant de tâche,
 l'utilisateur ne pourrait pas surcharger la méthode car deux méthodes avec le même nom mais
  des paramètres différent ne représenteraient qu'un seul nœud. 
\newline
\newline
\textbf{Exercice 3}
\newline
\textbf{2. Donner les grandes lignes de votre algorithme de la méthode execute. Il est attendu un
algorithme décrit de manière synthétique et non un copier/coller de votre code.}

\begin{algorithm}
  \caption{Algorithm Exercice 3}\label{euclid}
  \begin{algorithmic}[1]
    \Procedure{Execute}{$a,b$}
      \State $G\gets Graph<String>$
      \State $C\gets Context<String , Object>$
      \State $Mtab\gets Method[]$
      \State $result\gets Map<String , Object>$
      \For{\texttt{<node in Graph>}}\Comment{ont execute les method racine}
        \State \texttt{<Chercher la méthode qui correspond au node>}      
        \If{$nodeParam$ is in $C$ or $nodeNbParam$ == 0}
          \State $result\gets resultat$
          \ElsIf{si il y au moins une dépendance qui n'est pas dans C}
          \State $Mtab\gets methode$\Comment{la méthode de node}
        \Else 
        \State \texttt{<invoke methode de node en recupérant les parametre dans C>}
        \State \texttt{<ou sans parametre>}
        \EndIf  
        \EndFor
        \State 
      \For{\texttt{<m in Mtab>}} \Comment{Ont execute les méthode qui présenter des dépendance}
        \State \texttt{<invoke m en recupérant les parametre dans G et C>}
      \EndFor
      \State \textbf{return} $result$
    \EndProcedure
  \end{algorithmic}
\end{algorithm}

Pour résumer l'algorithme utilisé, au debut en récupérer tous les résultats des tâches racine tout en gardant les méthodes 
des autres tâches qui ne peuvent pas encore être exécutées, puis parcourir les méthodes restantes et les exécuter.
\newline
\newline
\textbf{Exercice 4}
\newline
\textbf{1. Décrire comment vous avez assuré le parallélisme des tâches indépendantes.}
\newline
\newline
Nous avons considéré chaque nœud du graphe et lui avons dédié un thread. Il commence par chercher la 
méthode qu'il doit appeler, puis pour chaqu'un de ses arguments va chercher ça valeur ou dans le context ou la map resultat, si ce dernier n'existe pas encore dans la map, 
il attend Une fois qu'il a fini de recupérer la valeur de arguments, il invoque la fonction et met le resultat dans la map
\newline
\newline
\textbf{Exercice 5}
\newline
\textbf{1. Décrire votre protocole de communication}
\newline
Le JobExecutorRemote demande demande le calcule d'un job a une jvm distante puis il ce mets on attente
que celle ci lui renvois la Map String , Object  calculer.
une fois fait il demande sont resultat
\newline
\textbf{2. Justifier votre choix de l’API de communication que vous avez utilisée.}
\newline
Nous avons choisi Rmi car elle paraissait être la plus adaptée au problème. Nous n'aurons pas besoin de gérer une base de donnée,
il faudra juste gérer l'execution de tâche sur une ou plusieurs Jvm, elle est aussi incluse dans java donc pas besoin d'installation préalable ce qui est un plus.
cependant le seul default de l'api est la notification du client qui ne peux etre faite qui si ce dernier est Remote
\newline
\newline
\textbf{3. Quelle hypothèse doit-on faire sur le type des objets du contexte ? Justifiez.}
\newline
L'hypothèse serait que les types utilisés dans "Object" sont sérializable sinon on ne peut pas les 
transmettre d'une machine à une autre.
\newline
\newline
\textbf{4. Quel mécanisme avez-vous utilisé pour la notification d’avancement au client ?}
\newline
\newline
Sur le serveur a chaque tache demander , une nouvelle entrée dans la map est crée avec le resultat ainsi que le nombre de tache terminer par Job
pour connaitre l'avancement de la tache le client demande au serveur qui va lui rendre le nombre de tache terminer
\newline
\newline
\textbf{Exercice 6}
\newline
\textbf{1. Quel est le protocole de communications entre maître-esclaves et éventuellement entre
esclave-esclave ?}
\newline
1-le Maitre reçoit la demande à qui il affecte un id et le renvoi au client 
\newline
2- le client demande son résultat et se mets en attente si son résultat n'est pas encore prêt 
\newline
3-le maitre lance un thread qui va chercher  des esclaves à qui il demandera d'exécuter une tâche si possible 
\newline
4-une fois trouvé, il lui passe le nom des tâches qu'il doit exécuter ainsi que le Job
\newline
5-une fois que l'esclave a terminé, il met la valeur de retour sur Taskmaster et décrémente un cpt dès que celui-ci est à 0 le client est libre de récupérer son Job
\newline
\newline
\textbf{2. Comment le maître affecte ses tâches équitablement sur les esclaves ?}
\newline
À chaque demande, il déplace l'index du premier esclave qui va recevoir la tâche ainsi il effectue une rotation complete et les thread qui demande 
les tache font de meme.
\newline
\textbf{3. Comment les esclaves sont assurés de respecter leur borne de tâche courante ?}
\newline
pour chaque tâche, l'esclave lance un Thread : Au début il decrémente la variable maxtask et la fin de son exécution il l'incrémente
\newline
\textbf{4. Comment gérez-vous la communication d’un résultat entre deux tâches dépendantes ? Don-
ner les avantages et les inconvénients de votre solution.}
\newline
Les différentes tâches communiquent par le billet d'une map stockée sur le maitre. 
À chaque fois qu'une tâche se termine, elle dépose sa valeur de retour sur le maître, à l'inverse, si elle a besoin d'une valeur, il va l'y récupérer. 
\newline
\textbf{5. (niveau 2) Comment gérez-vous le fait d’avoir plusieurs jobs en cours d’exécution sur le
cluster ?}
\newline
le master ne fait que recevoir les jobs, le reste est fait par un thread pool
\newline
\textbf{6. (niveau 3) Décrire le mécanisme qui permet de détecter la panne d’un esclave. Comment
gérez-vous la réaffectation des tâches perdues ?}
\newline
la panne d'un esclave est détectée lors de l'affectation des tâches, si l'esclave renvoi une connection exception c'est qu'il est en panne 
et par conséquent considéré par le thread qui affect les tâches comme tel pour le reste de l'exécution de ce thread.
Les esclave qui avaient reçu un job reçoivent un signal d'abondon et on redemande toutes les tâches qui ne sont pas présentes dans map de resultat.

\end{document}
%\begin{algorithm}
%  \caption{Algorithm Exercice 3}\label{euclid}
%  \begin{algorithmic}[1]
%    \Procedure{Execute}{$a,b$}\Comment{The g.c.d. of a and b}
%      \State $r\gets a\bmod b$
%      \While{$r\not=0$}\Comment{We have the answer if r is 0}
%        \State $a\gets b$
%        \State $b\gets r$
%        \State $r\gets a\bmod b$
%      \EndWhile\label{euclidendwhile}
%      \For{\texttt{<some condition>}}
%        \State \texttt{<do stuff>}
%      \EndFor
%      \State \textbf{return} $b$\Comment{The gcd is b}
%    \EndProcedure
%  \end{algorithmic}
%\end{algorithm}